\documentclass[12pt, a4paper]{report}
\usepackage[a4paper, total={6in, 10in}]{geometry}

\usepackage[portuges]{babel}
\usepackage[utf8]{inputenc}
\usepackage{graphicx}
\usepackage{url}
\usepackage{enumerate}
\usepackage{xspace}


% Documento
\begin{document}

\title{
    Fundamentos de Sistemas Distribuídos\\
    \textbf{\\Trabalho Prático}
    \large{\\Relatório de Desenvolvimento}
}

\author{
    Miguel Oliveira\\ pg41088
    \and Pedro Moura\\ pg41094
    \and César Silva\\ pg41842
}
\date{Universidade do Minho,\\\today}

\maketitle

\begin{abstract}
    Este relatório descreve o desenvolvimento de um projeto no âmbito da UC de Fundamentos de Sistemas Distribuídos, onde, através de ferramentas e bibliotecas apresentadas nas aulas, são aplicadas técnicas para um bom e, sobretudo, fiável funcionamento de um sistema distribuído.
\end{abstract}

\tableofcontents

\chapter{Introdução}
Com o objetivo de aplicar os conhecimentos adquiridos nas aulas relativas à UC de Fundamentos de Sistemas Distribuídos, foi-nos proposto o desenvolvimento de um sistema de troca de mensagens com persistência e ordenação, inspirado na rede social \textit{Twitter}.
O enunciado apresenta alguns \underline{requisitos} que devem ser respeitados:
\begin{enumerate}
    \item O sistema deve incluir um conjunto de servidores, que se conhecem todos entre si. Admite-se a possibilidade de um destes servidores ser reiniciado, devendo garantir que o sistema continua a funcionar depois de todos os servidores estarem novamente operacionais. O servidor não deve ter qualquer interação direta com o utilizador.
    \item O sistema deve incluir clientes que se ligam a qualquer um dos servidores. Admite-se que o cliente pode ser reiniciado e ligado a um novo servidor. O cliente deve incluir uma interface rudimentar para interagir com o sistema, que deve ter algumas funcionalidades.
    \item Admite-se que tanto os clientes como os servidores podem fazer uso da memória persistente.
    \item O conjunto de mensagens obtido por cada cliente em cada operação deve refletir uma visão causalmente coerente das operações realizadas em todo o sistema, por esse ou outros utilizadores.
\end{enumerate}

Neste primeiro capítulo foi feita a contextualização e foram apresentados os objetivos deste projeto.
De seguida, apresentamos uma proposta de solução, onde são descritas tanto as abordagens seguidas bem como as decisões tomadas para atingirmos os objetivos.
No terceiro capítulo, descrevemos como chegamos à proposta de solução apresentada, com detalhes mais técnicos.
Posto isto, no quarto capítulo, apresentamos o resultado final da aplicação acompanhados com alguns testes realizados.
Por fim, no quinto e último capítulo, sumarizamos o que foi escrito no relatório, e a nossa satisfação global com o projeto.


\chapter{Proposta de Solução}
\section{Servidores}




\begin{itemize}
    \item Arquitetura do Sistema
\end{itemize}


\chapter{Desenvolvimento}
\begin{itemize}
    \item Tecnologia
    Falar do java, intellij e atomix
    \item Implementação
    
\end{itemize}


\chapter{Resultado}
\begin{itemize}
    \item Testes
    \item Resultado
\end{itemize}


\chapter{Conclusão}
\begin{itemize}
    \item Discussão dos resultados
    \item Pros/contras da abordagem seguida
    \item Como poderíamos evoluir o que foi construído
    \item Sempre com enfoque que foi feito tudo o que foi pedido
\end{itemize}


\end{document}
