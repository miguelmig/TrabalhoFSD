\documentclass[12pt, a4paper]{report}
\usepackage[a4paper, total={6in, 10in}]{geometry}

\usepackage[portuges]{babel}
\usepackage[utf8]{inputenc}
\usepackage{graphicx}
\usepackage{url}
\usepackage{enumerate}
\usepackage{xspace}


% Documento
\begin{document}

\title{
    Fundamentos de Sistemas Distribuídos\\
    \textbf{\\Trabalho Prático}
    \large{\\Relatório de Desenvolvimento}
}

\author{
    Miguel Oliveira\\ pg41088
    \and Pedro Moura\\ pg41094
    \and César Silva\\ pg41842
}
\date{Universidade do Minho,\\\today}

\maketitle

\begin{abstract}
    Este relatório descreve o desenvolvimento de um projeto no âmbito da UC de Fundamentos de Sistemas Distribuídos, onde, através de ferramentas e bibliotecas apresentadas nas aulas, são aplicadas técnicas para um bom e, sobretudo, fiável funcionamento de um sistema distribuído.
\end{abstract}

\tableofcontents


\chapter{Introdução}
Com o objetivo de aplicar os conhecimentos adquiridos nas aulas relativas à UC de Fundamentos de Sistemas Distribuídos, foi-nos proposto o desenvolvimento de um sistema de troca de mensagens com persistência e ordenação, inspirado na rede social \textit{Twitter}.

\begin{itemize}
    \item Contexto - done
    \item Motivação
    \item Objetivos
    \item Organização do relatório
\end{itemize}


\chapter{Enunciado}


\chapter{Trabalhos Relacionados}


\chapter{Proposta de Solução}
\begin{itemize}
    \item Arquitetura do Sistema
\end{itemize}


\chapter{Desenvolvimento}
\begin{itemize}
    \item Tecnologia
    \item Implementação
\end{itemize}


\chapter{Resultado}
\begin{itemize}
    \item Testes
    \item Resultado
\end{itemize}


\chapter{Conclusão}
\begin{itemize}
    \item Discussão dos resultados
    \item Pros/contras da abordagem seguida
    \item Como poderíamos evoluir o que foi construído
    \item Sempre com enfoque que foi feito tudo o que foi pedido
\end{itemize}


\end{document}
